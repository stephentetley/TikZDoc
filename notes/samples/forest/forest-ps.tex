% For Postscript
\documentclass{article}


\usepackage{forest}


%% Copied from the doc this should be includeable as a library somehow
%% Check source...
\forestset{
    declare dimen register=folder indent,
    folder indent=.45em,
    folder/.style={
        parent anchor=-children last,
        anchor=parent first,
        calign=child,
        calign primary child=1,
        for children={
            child anchor=parent,
            anchor=parent first,
            edge={rotate/.option=!parent.grow},
            edge path'/.expanded={
                ([xshift=\forestregister{folder indent}]!u.parent anchor) |- (.child anchor)
            },
        },
        after packing node={
                if n children=0{}{
                tempdiml=l_sep()-l("!1"),
                tempdims={-abs(max_s("","")-min_s("",""))-s_sep()},
                for children={
                    l+=tempdiml,
                    s+=tempdims()*(reversed()-0.5)*2,
                },
            },
        },
    }
}


\begin{document}
\begin{forest}
    for tree={grow'=0,folder,draw}
    [factx-fsharp
        [src
            [bin, node options={fill=red!25}]
            [FactX
                [Internal, node options={fill=green!25}
                    [Common.fs, node options={fill=green!25}]
                ]
                [FactOutput.fs]
                [FactWriter.fs]
                [Pretty.fs]
                [Syntax.fs]
            ]
            [obj]
        ]
        [test
            [FactXTest.fsproj]
            [Test01.fsx]
        ]
    ]
\end{forest}
\end{document}
